
\subsection{Switched Inductor}\label{ch:SIBC}
\todo[color=c04b,inline]{testing the colors}
The Switched Inductor Boost Converter (SIBC) is a small alternation to the Conventional Boost Converter mentioned in Chapter \ref{ch:CBC}.
It only adds passive components to an already known topology,
which means the same switching scheme can be used.
\subsubsection{Additions from Conventional BC}
In the SIBC,
the single inductor from the Conventional BC gets replaced by two inductors and three diodes,
connected as shown in Figure \ref{fig:SI}.

\begin{figure}[H]
   \centering
   \includegraphics[width=\textwidth]{figures/bSwitchedInductor/switched_inductor.pdf}
    \caption{Switched inductor boost converter circuit}
	\label{fig:SwitchedInductor}
\end{figure}


With this configuration,
the inductors form two different topologies during the ON and OFF states. By reversing the biasing of the diodes the inductors are connected in parallel during ON state and series during OFF state.  \todo[color=c04b]{refer to figure maybe} 


\subsubsection{Switching States}
The same switching scheme as for the Conventional BC can be used,
as mentioned earlier.
The equivalent circuits during the on and off stage are shown in Figure \ref{fig:OnOffSIBC}.

\begin{figure}%
    \centering
    \subfloat[Switch ON]{{\includegraphics[width=6.5cm]{figures/bSwitchedInductor/switched_inductorON.pdf} }}%
    \qquad
    \subfloat[Switch OFF]{{\includegraphics[width=6.5cm]{figures/bSwitchedInductor/switched_inductorOFF.pdf} }}%
    \caption{Switching states of the SIBC}%
    \label{fig:example}%
\end{figure}

some text
\todo[color=c04b,inline]{calculations}
some text
\todo[color=c04b,inline]{simulations}
some text

As you can see in Chapter \ref{ch:dataAcq}

If you want to write a paragraph on something,
this will still be the same line.
I will make this paragraph a little bittle longer,
so that it actually uses more than one line.
I think it does now.

New paragraphs start with a proceeding empty line in the editor.