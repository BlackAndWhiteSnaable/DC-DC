
\subsection{Switched Inductor}\label{ch:SIBC}
\todo[color=c04b,inline]{testing the colors}
The Switched Inductor Boost Converter (SIBC) is a small alternation to the Conventional Boost Converter mentioned in Chapter \ref{ch:CBC}.
It only adds passive components to an already known topology,
which means the same switching scheme can be used.
\subsubsection{Additions from Conventional BC}
In the SIBC,
the single inductor from the Conventional BC gets replaced by two inductors and three diodes,
connected as shown in Figure \ref{fig:SI}.

\missingfigure{switched Inductor} \label{fig:SI}

With this configuration,
both inductors get charged up during the on stage
and can discharge into the output capacitor during the off stage.
The diodes prevent the current to run back into the DC link \todo[color=c04b]{maybe rewrite this}
thus preventing damage.


\subsubsection{Switching States}
The same switching scheme as for the Conventional BC can be used,
as mentioned earlier.
The equivalent circuits during the on and off stage are shown in Figure \ref{fig:OnOffSIBC}.

\missingfigure{On and off stage equivalents of the SIBC}\label{fig:OnOffSIBC}

some text
\todo[color=c04b,inline]{calculations}
some text
\todo[color=c04b,inline]{simulations}
some text

As you can see in Chapter \ref{ch:dataAcq}

If you want to write a paragraph on something,
this will still be the same line.
I will make this paragraph a little bittle longer,
so that it actually uses more than one line.
I think it does now.

New paragraphs start with a proceeding empty line in the editor.