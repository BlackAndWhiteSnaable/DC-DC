\section{Component Specifications}\label{ch:compSpec}

\subsection{MOSFETs}
Compared to IGBTs MOSFETs are better for applications with high current pulses and high frequencies. Because the simulations had shown high peak currents over the switch the MOSFET was chosen.
The specific MOSFET is the C2M0160120D from Wolfspeed/Cree. 
It features a high blocking voltage, low on-resistance and low capacity.
That means it is able to work efficiently in high frequency power applications with small losses in switching, making it very suitable for converters.

The MOSFET is rated for switching voltages up to 1200V with a maximum current of 19A.
For switching a gate voltage of 14V is required and a negative input of -5V to -10V as a low input is recommended for fast switching.

% https://www.wolfspeed.com/media/downloads/169/C2M0160120D.pdf
\todo[color=c05a,inline]{Ref}

\subsection{Diodes}
The used diode is the STTH6012 from STMicroelectronics.
The ultra fast soft recovery characteristics makes it very efficient in high frequency pulse operation.
With a rating of 1200V repetitive peak reverse voltage and a maximum forward current of 60A the diode seems oversized but for experimental work they were used.

% https://www.st.com/resource/en/datasheet/stth6012.pdf
\todo[color=c05a,inline]{Ref}


\subsection{Capacitors}
The chosen capacitors are standard sizes of $2.2 \mu F$ capacity with a voltage rating to 450V.

\subsection{Inductors}
%The inductor was winded by hand over a
\todo[color=c05a,inline]{missing}