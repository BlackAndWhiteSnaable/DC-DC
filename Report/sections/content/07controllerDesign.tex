\chapter{Controller Design}\label{ch:controldesign}
There exist many different approaches to designing a controller,
all of which have their advantages and disadvantages.
For this project we decided to use a very simple approach,
the Proportional Derivative Integral (PID) controller.
This chapter compares this to another possible option,
the state-space controller, explains both of them briefly
and argues for our choice.

\section{PID control}\label{sec:PID}
A PID controller consists of three parts,
a proportional, integral and derivative part,
which it inherits its name from.
A mathematical representation can be given by one of the two equations in Equation \ref{eq:PID}.

\begin{equation}
	  D_{cl}(s) = k_P + \frac{k_I}{s} + k_D s$$
	$$D_{cl}(s) = k_P ( \frac{T_I}{s} + T_D s)
	 \label{eq:PID}
\end{equation} \cite{Franklin2014}\\
These equations can be visualised in block diagrams, as shown in Figure \ref{fig:twoPID}.
%\begin{figure}[H]
%    \centering
%    \includegraphics[width=\textwidth]{figures/07controllerDesign/PID_explicit.pdf}
%    \caption{Comparison of two common PID designs}
%	\label{fig:twoPID}
%\end{figure}

Figure \ref{fig:twoPID} also shows the typical placement of a PID-controller,
on the forward loop, behind the subtraction of the feedback and before the input to the plant.

The gains $k_P$, $k_I$ and $k_D$ have to be tuned according to the dynamics of the plant
and the desired characteristics of the controller.
This tuning can be done manually, or methodically.
We chose to use the Ziegler-Nichols tuning method,
as described by Franklin et al. \cite{Franklin2014},
as this method promised results close to our requirements and was discussed in our courses.

To use this method, some knowledge about the system is required.
If the system reacts stable to a step input,
the output measured can be used to tune $k_P$, $k_I$ and $k_D$
according to the Quarter Decay Method (QDM) \cite{Franklin2014}.

In the case that the system has an unstable reaction,
the Ultimate Sensitivity Method (USM) should be used.
Here the $k_p$ is increased until the system becomes marginally stable.
\cite{Franklin2014}

In both cases the output of the system is to be graphed over time
and from the corresponding graph some characteristic values can be read.
Because we are using the quarter decay method,
we won't discuss the ultimate sensitivity method any further.

As mentioned before,
to use the QDM, we need a step response of the system.
The following section explains step response in more detail and shows our analysis.

\subsection{Quarter Decay Tuning}\label{sub:quadec}
To obtain all needed data, the step response of the system
needs to be analysed.

\subsubsection{Step Response}
The step response needs to be measured on the Open Loop (OL) system.
A block diagram of an OL setup can be seen in Figure \ref{fig:OL}.

%\begin{figure}[H]
%    \centering
%    \includegraphics[width=0.75\textwidth]{figures/07controllerDesign/OLblock.pdf}
%    \caption{OL block diagram of a system}
%	 \label{fig:OL}
%\end{figure}

The recorded output can be seen in Figure \ref{fig:stepin}.
The figure also shows two red lines, approximating the slope and the final value.

%\begin{figure}[H]
%    \centering
%    \includegraphics[width=\textwidth]{figures/07controllerDesign/StepResponseLabeled.eps}
%    \caption{response to a step input with value 10}
%	\label{fig:stepin}
%\end{figure}

From this graph we can approximate the slope $R$ and the Lag $L=t_d$.\\
Because we are scaling the $\omega P$ down by a factor of 10, so we can directly input a percentage,
we had to scale the aforementioned unit step up by a factor of 10,
in order to get usable results.
While encountering this problem, we also noticed, that the pumps don't spin at $\omega \leq 9\%$.
When using the corrected step input, we got the measurements shown in Figure \ref{fig:stepin}.

Our analysis of figure \ref{fig:stepin} gives us the following values:
\\
\begin{tabular}{r c l l}
	$R$ 	& $=$ & $2.0236$ 	& \footnotesize{\textit{slope}}\\
	$t_d=L$	& $=$ & $0.75$ 		& \footnotesize{\textit{lag}}\\
\end{tabular}


\subsubsection{Tuning}
The QDM is tuning a P, PI or PID controller $D_c(s)$ with the formula shown in the second Equation \ref{eq:PID}.
The values of $k_P$, $T_I$ and $T_D$ are scalar gains,
tuned according to the characteristics obtained from figure \ref{fig:stepin}.
\\
\begin{center}
\begin{tabular}{l r c l l}
	Type of Controller	& \multicolumn{3}{ c }{Optimum Gain}\\
	\hline
	\multirow{1}{*}{P}	& $k_P$ & $=$ & $\nicefrac{1}{RL}$	\\
	\\
	\multirow{2}{*}{PI}	& $k_P$ & $=$ & $\nicefrac{0.9}{RL}$\\
						& $T_I$ & $=$ & $\nicefrac{L}{0.3}$	\\
	\\
	\multirow{3}{*}{PID}& $k_P$ & $=$ & $\nicefrac{1.2}{RL}$\\
						& $T_I$ & $=$ & $2L$				\\
						& $T_D$ & $=$ & $0.5L$ 				\\
\end{tabular}
\end{center}
Note that the parameters $T_I$ and $T_D$ are not mentioned for the first two controller types,
as they are to be set to 0.

\subsubsection{Results}
We tested all three options of P, PI and PID control with the tuned parameters,
the results can be seen in Figure \ref{fig:PIDtest}.
Please note, that the P-controller with the calculated value did not achieve any results,
as the error was too small to start the pump.
To overcome this issue, we multiplied $k_P$ with 10.
This was solely done for the P-controller,
because the PI- and PID-controller gave proper results without this correction.
We expect this to be caused by the scaling factor before the input to the pumps,
as discussed in Subsection \ref{sub:quadec}.


%\begin{figure}[H]
%    \centering
%    \includegraphics[width=\textwidth]{figures/07controllerDesign/Ptest.eps}
%    \includegraphics[width=\textwidth]{figures/07controllerDesign/PItest.eps}
%    \includegraphics[width=\textwidth]{figures/07controllerDesign/PIDtest.eps}
%    \caption{Results from testing a P, PI and PID controller with ZN tuning}
%	\label{fig:PIDtest}
%\end{figure}

As was expected, the P-controller settles the fastest,
at approximately 4 seconds,
while the PI and PID settle around 20 seconds.
The P-controller has a very big steady-state error though,
which doesn't fulfil our requirements.
The PI-controller does not have a steady-state error,
but the overshoot of approximately 25\% doesn't fulfil our requirements either.
Interestingly, the PID-controller also has an overshoot above our limits,
which we expect to be caused by the ZN tuning.
In the next step we tried manually tuning the $k_D$ coefficient,
to get an overshoot inside our requirements.

\subsection{Manual Tuning}\label{sub:manualPID}
Building on top of the best candidate, the PID-controller,
we manually tuned the coefficients.
We came to the conclusion,
that the overshoot is happening, because the controller is too aggressive,
which is why we decided to decrease the $k_P$ coefficient to $0.2$.
Looking back at the second formula in Equation \ref{eq:PID},
we can see, that this will not affect the relations between the coefficients.

\subsubsection{Results}
The results of the manual tuning can be seen in Figure \ref{fig:manualPID}.

%\begin{figure}[H]
%    \centering
%    \includegraphics[width=\textwidth]{figures/07controllerDesign/manualPID.eps}
%    \caption{Results from ZN PID with $k_P = 0.2$}
%	\label{fig:manualPID}
%\end{figure}
Interesting to notice here is,
that the controller has a deadtime of approximately 4 seconds,
before its output affects the system.


One big disadvantage of PID tuning for this system is,
that it is not taking advantage of the existence of three pumps as  individual inputs.
The tuning in this chapter is based on using only a single pump,
but could easily be repeated for multiple pumps with equal input speed.
Based on research done by Pedersen and Yang \cite{YangMultiPump2008},
it seems that using multiple pumps could increase efficiency,
but it be most efficient to always spin all used pumps equally.
This is not possible with a single PID controller,
as it only outputs one speed to all connected pumps.
It could be possible to implement an additional logic,
switching between differently tuned controllers for different flow requirements.